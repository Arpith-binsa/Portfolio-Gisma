\documentclass[12pt]{article}
\usepackage[margin=1in]{geometry}
\usepackage{hyperref}
\usepackage{enumitem}

\title{\textbf{Portfolio Project Report}}
\author{Arpith Binsa \\ GH1030804 \\ B201A Computer Science Lab (SS0325)}
\date{2-july-25}

\begin{document}

\maketitle

\section*{Website Details}
The website was built on top of a template from bootstrap. But the finished product has been fully customized and only resemble the template just in terms of design choices.
I went with bootstrap instead of the many other choices mainly because it was recommended by the prof but also because I found the templates simplistic and easy to modify and the source code of the template was easy to understand.

I bought a personal custom Domain \href{https://www.arpithbinsa.com} {(www.arpithbinsa.com)} as I intend to use this website in the future as my personal portfolio and blog. The website was bought from \href{www.goddady.com}{godaddy.com} and is monitored using \href{www.cloudfare.com}{cloudfare.com}. It was coded on Vscode with an extension called live server which lets me see the changes I make to the code in Realtime as a website.

The website is hosted on a free GitHub page as its known for being reliable, open-source, trustworthy and feature-rich. After 	Pushing the git and the website was live, several other updates were made to the website from GitHub. A new Branch was made and titles Test branch, here the update codes were tested and only after that was the pull request made and the code pushed into the main branch and then committed. This shows that I have a good knowledge on how to work on git efficiently and in a professional manner.

The website itself is very simple without any complex design, transition algorithms etc. It was made using HTML and CSS as these were the only tools required to make this simple website. The website uses a tool called “Form Submit” from Devrolabs.com to send the filled-out form on my website to my personal email address.

\section*{Site Structure}
The website has 6 main sections.

\begin{enumerate}[label=\arabic*.]
    \item Title card
    \item About me section
    \item My work section
    \item LinkedIn link section
    \item CV
    \item Contact form
\end{enumerate}

\section*{Title}
The title is just there to grab attention from visitors to my website future employers, colleagues etc., it gives a very short intro on who I am, and it’s written in huge bold letter to capture attention.
\section*{About Me}
The about me contains more information on my interest in all thing’s computer science, how I got into this field, some of my past projects and a brief intro to what programming languages I am capable of working in.

\section*{My Work}
This section contains for different subtitles and some projects I handled in each title, the titles are 'PC builds', 'Cloud and network setups', 'Cloud infrastructure (AWS)' and 'System diagnostics and repairs'. Each of these titles are followed up by a short view on what projects I have contributed to this topic, and it also shows some thumbnails, pictures, and details on the projects I have mentioned. This part of the website was the hardest to do and took the most amount of time as I hadn’t used HTML in a while and was still a little rusty. I Uploaded the images, thumbnails and made sure all of them were the same dimensions and each of them were correctly assigned. This took the most time as I had to crop each image multiple times to make sure they fit perfectly.
\begin{itemize}
    \item PC Builds
    \item Cloud and Network Setups
    \item Cloud Infrastructure
    \item System Diagnostics and Repairs
\end{itemize}

This took the most time as I had to crop each image multiple times to make sure they fit perfectly.

\section*{CV}
My CV sections contains a downloadable PDF of the CV and a source code tex file that contains the latex code. I simply copied both of these files to the assets folder and linked them.

\section*{LinkedIn}
This section is just an interactable link to LinkedIn page, and I got the LinkedIn logo for the button from the Bootstrap Icons website directly and it was easy to integrate into my website as I just need to write the Icon name in html and no need to upload any files.

\section*{Contact Form}
This was the most technical part of the website as I needed a cloud-based software that can read the form submitted and send them to my Gmail account. From many options I chose “formsubmit” from Devrolabs as it was super easy to setup and completely free of use until the limit of 100 forms per month has reached. It was a bit difficult to integrate this into the html code as I had never used this software before, but with a little help from an online tutorial I was able to do this and test the form withing hours.

OUTSIDE HELP

\section*{Outside Help}
As this was an important project that not only dictated my grades but would also come in handy with my future endeavors I had to use some outside help. 
The use of Large Language Models like chat gpt was kept to a minimum as was only ever used when I couldn’t understand a concept from some of the new software’s, it was never used directly to write code or form paragraphs.

However, I had to get help when hosting this website on GitHub as I was not familiar with hosting on GitHub and eventually ran into problems trying to connect the custom domain to my GitHub page. So, I had to seek help with my elder brother, who works at a Startup company as a Front-End Developer. His help was critical to this project as I ran into more issues with GoDaddy when trying to host the website.

\section*{Conclusion}
Developing an online portfolio has been on Wishlist for a long time but I never had a need to do it until now. And I’m grateful to the prof for this opportunity as I’m sure this will help me in the future with a competitive field such as IT. It allowed me to showcase all of my work which I though was not significant enough to post on my LinkedIn page or social media. This will be used both as a portfolio and a blog in the furture.
\section*{Live Site Link}
\href{https://www.arpithbinsa.com}{https://www.arpithbinsa.com}

\end{document}
